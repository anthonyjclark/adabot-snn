
\section{Training a Simulator Neural Network}

The goal of our study is to develop a SNN that is able to determine a new pose for our mobile robot based on the input control signals. To train the neural network, we first need some way to collect training data.

\noindent
\textbf{Collecting training data.}
%
We built a physical simulation of our mobile robot using DART~\citep{Lee.2018.JOSS.DARTDynamicAnimation}.
%
The simulation can be run with different values for the robot's input control signals: the speeds of the left and right wheels as well as the strut extension amount. The simulation did not include any obstacles or
%
We ran the simulation with 9012 different combinations of these input signals and collected the resulting change in position and heading.

\noindent
\textbf{Training the SNN.}




noisy measurements
uneven terrain





\noindent
\textbf{Comparing the SNN model.}

running times (phys vs snn)
7.5s
.28


difference between SNN and diff will be exaerbated


One example. Others are more or less dramatic

evolve then show behavior in phys vs differential drive model






\noindent
\textbf{Evolving a simple controller.}

2 minutes vs 1 hour


Use cases:
(1) module onboard
(2) evolve high-level strategy control parameters

\citet{Pretorius.2014.2ICECC.ComparisonNeuralNetworks}
more efficient (30$\times$)

\noindent
\textbf{Evolving a simple controller.}



future
- obstacles
- uneven terrain
